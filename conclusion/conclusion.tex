To conclude, this project has made significant strides in enhancing the SST
turbulence model for optimization purposes within ADflow. The following key
points highlight the project's achievements:

First, the distance to the nearest wall had to be extended for halo cells. This
was necessary in order to get rid of some intermediate MPI communication in the
$F_1$ blending function of SST. It was important to get rid of this
communication as it prevents automatic differentiation.

The implementation of Automatic Differentiation (AD) for the SST turbulence
model was a major milestone. This implementation facilitated the efficient
computation of partial derivatives, which are essential for optimization tasks
in ADflow. But also the regular solvers like \textit{Approximate Netwon-Krylov
(ANK)} and \textit{Newton-Krylov (NK)} profit from this as they now have access
to an AD-preconditioner.

It has been shown that the changes did not change the legacy implementation of
SST. Using testcases from NASA's \textit{Turbulence Modeling Resource (TMR)},
it has also been shown that the results obtained mostly agree to reference data.
Although there is still some uncertainty because a different production term for
the turbulence had to be used in order do achieve convergence. It was also not
possible to obtain results for the \textit{flatplate} testcase. The author is
confident that a bit more time would adress this shortcomings.

It is even possible to obtain total gradients using the adjoint method when
assembling the state residual matrix using forward routines. The aerodynamic
design variables appear to be correct, although to relatively low tolerance.
This my be explained through the high non-linearity of SST. It is also
important to note that the geometric design variables seem to be wrong.
Additionally, they appear to depend on the number of cpus used. This most
likely indicates a problem with the extension of the wall distance to halo
cells.

When trying to solve the adjoint system using the reverse\_fast routines, on
may get a number. But it is completely wrong. But the author still considers
this an achievement because ADflow does not simply crash.

In order to maintain the quality and integrity of ADflow, regression test are
used. This project did extended them for the SST turbulence model. But because
the implementation of SST is only at a prototype stage, the tests are as well.

\section{Comparison to Goals}
When looking at the goals stated in the introduction (section \ref{sec:goals}),
most of them have been partially achieved. The following lines take a look at
the goals that were missed or only partially fulfilled.

\begin{enumerate}

    \item[4.] Make sure the partial AD derivatives are correct. 
        \begin{itemize}
            \item This has almost been achieved. Only the backwards mode for
                the Forces is wrong. Also, the backwards\_fast mode is not
                correct.
        \end{itemize}

    \item[5.] Get the NK/ANK Solver working.
        \begin{itemize}
            \item Apart from the finite differences pre-conditioner, this has
                been achieved.

            \item The physicality check probably also needs some adjustment for
                SST.
        \end{itemize}

    \item[6.] Test and verify the implementation
        \begin{itemize}
            \item This has been partially achieved. It would be great if the
                flatplate testcase would have worked. Also the different
                production terms for SST need to be tested more.
        \end{itemize}

    \item[7.] Get the adjoint Solver working.
        \begin{itemize}
            \item Has been achieved in the sense that it does not crash 
        \end{itemize}

    \item[8.] Test and verify the total gradients.
        \begin{itemize}
            \item This has been partially achieved. The aerodynamic design
                variables appear to be correct when using forward AD for the
                state residual matrix. But the aerodynamic variables are
                slightly off. 

            \item The gradients are completely off when using the
                backwards\_fast routines.
        \end{itemize}
\end{enumerate}

To conclude, SST has now achieved a prototype state where it is almost ready
for optimization. There are still some bugs present, but it is clear where they
are and mostly by what they are caused. The only thing that prevented the
author from fixing them was a lack of time. As such, one has to realize that
getting SST fully working for optimization is a big task. Thus, the outcome of
this project should be considered a success.



\section{Outlook}
This report stated openly and clearly where bugs are present and by what they
are most likely caused. As fixing those bugs and getting SST ready for
production is probably the most valuable target of a proceeding project, the
steps that come next should be clear. But still, the following line lists the
most prominent points in random order:

\begin{itemize}
    \item Adapt the physicality check in ANK-turb for SST

    \item Get the finite difference preconditioner working in ANK for SST.
        Maybe only the step size needs to be changed.

    \item Make sure the partials are correctly summed up in the reverse
        routines that exchange the wall distance in halos.

    \item Take a closer look at what is going on with total derivatives
        obtained through complex step. The results have shown that there is a
        slight difference between a step size of $1e-40$ and $1e-200$. This
        should not be the case.
        
    \item Make sure all the gradients are correct when solving the adjoit system
        using the forward AD routines.

    \item Fix the partial backwards\_fast derivatives.

    \item Make sure the gradients are correct when using the backwards\_fast
        routines.

    \item Optimize an airfoil using SST.
\end{itemize}

