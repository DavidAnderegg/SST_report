 % \documentclass[a4paper, draft]{report}
\documentclass[a4paper, 10pt]{report}


\usepackage[utf8]{inputenc}
\usepackage[T1]{fontenc}
\usepackage[margin=2.5cm]{geometry}
\usepackage{csquotes}
\usepackage{hyphenat}
\usepackage{biblatex}
\usepackage{graphicx}
\usepackage{float}
\usepackage{caption}
\usepackage{subfig}
\usepackage{textcomp}
\usepackage{gensymb}
\usepackage{booktabs}
\PassOptionsToPackage{hyphens}{url}\usepackage{hyperref}
\usepackage{listings}
\usepackage{lmodern}
\usepackage{amsmath}
\usepackage{amssymb}
\usepackage{siunitx}
\usepackage{enumitem}
\usepackage[acronym]{glossaries}
\usepackage{tabularx}
\usepackage{xltabular}
\usepackage{setspace}
\usepackage{tablefootnote}


\usepackage{tablefootnote}

% \usepackage[scale=1.0]{draftwatermark}

\addbibresource{biblio.bib}
\newacronym{ad}{AD}{Algorithmic/Automatic Differentiation}
\newacronym{sa}{SA}{Spalart Allmaras}
\newacronym{imes}{IMES}{Institute for mechanical systems}

\graphicspath{
    {img}
    {fundamentals/img}
    {methods/img}
    {results/img}
}
% \hyphenation{Mathe-matik wieder-gewinnen}

% use 1e-10 instead of 1 x 10^(-10)
\sisetup{output-exponent-marker=\ensuremath{\mathrm{e}}}



\title{Extension of the SA turbulence model for rough walls in ADflow}
\author{David Anderegg}

% Add Numbering to chapters
\makeatletter
\def\@makechapterhead#1{%
  \vspace*{50\p@}%
  {\parindent \z@ \raggedright \normalfont
    \interlinepenalty\@M
    \Huge\bfseries  \thechapter.\quad #1\par\nobreak
    \vskip 40\p@
  }}
\makeatother

\renewcommand{\arraystretch}{1.1}



\usepackage{fancyhdr}
\setlength{\headheight}{34pt}

\renewcommand{\headrulewidth}{0.5pt}
\renewcommand{\footrulewidth}{0.5pt}


\fancypagestyle{plain}{
    % header inputs
    \lhead{IMES}
    \chead{Extension of the SA turbulence model for rough walls in ADflow}
    \rhead{\includegraphics[width=1cm]{uni_header}}


    %footer inputs
    % \lfoot{Innosuisse 35172.1 IP-ENG}
    \cfoot{Page  \thepage}
    \rfoot{\today}
}
\pagestyle{plain}

\setlength{\parskip}{.25em}

\begin{document}
    \begin{titlepage}

\newcommand{\HRule}{\rule{\linewidth}{0.5mm}} % Defines a new command for the horizontal lines, change thickness here

\center % Center everything on the page
 
%----------------------------------------------------------------------------------------
%   HEADING SECTIONS
%----------------------------------------------------------------------------------------

\textsc{\LARGE Zurich University of Applied Sciences}\\[1.5cm] % Name of your university/college
\textsc{\Large Specialisation Project 1}\\[0.5cm] % Major heading such as course name
\textsc{\large Institute of Mechanical Systems}\\[0.5cm] % Minor heading such as course title

%----------------------------------------------------------------------------------------
%   TITLE SECTION
%----------------------------------------------------------------------------------------

\HRule 
{ \huge \bfseries
  \begin{spacing}{1.2}
    Extension of the SA turbulence model for rough walls in ADflow
  \end{spacing}}
\HRule \\[0.5cm]
 
%----------------------------------------------------------------------------------------
% AUTHOR SECTION
%----------------------------------------------------------------------------------------

\begin{minipage}{0.4\textwidth}
\begin{flushleft} \large
\emph{Author:}\\
David Anderegg\\
%\secondauthorfirstname \textsc{\secondauthorlastname}% Your name
\end{flushleft}
\end{minipage}
~
\begin{minipage}{0.4\textwidth}
\begin{flushright} \large
\emph{Supervisor:} \\
Prof. Marcello Righi \\
Dr. Anil Yildirim (Michigan)
\end{flushright}
\end{minipage}\\[1.5cm]

% If you don't want a supervisor, uncomment the two lines below and remove the section above
%\Large \emph{Author:}\\
%John \textsc{Smith}\\[3cm] % Your name

%----------------------------------------------------------------------------------------
% DATE SECTION
%----------------------------------------------------------------------------------------

{\large \today}\\[2cm] % Date, change the \today to a set date if you want to be precise

%----------------------------------------------------------------------------------------
% LOGO SECTION
%----------------------------------------------------------------------------------------

\includegraphics[width=0.5\textwidth]{img/en-zhaw-imes-sw.png}\\[2cm] % Include a department/university logo

 
%----------------------------------------------------------------------------------------

\vfill % Fill the rest of the page with whitespace
% \pagenumbering{alph}

\end{titlepage}



    \begin{abstract}
      As the available computer power increases, Reynolds Averaged Navier Stokes
(RANS) based optimizations come more and more in reach. When performing such
high fidelity optimization, it is necessary to properly represent all flow
conditions. If one fails to do so, the optimizer might exploit effects that do
not exist in reality.

      This work extends the Spalart Allmaras (SA) turbulence model in the open
source CFD solver ADflow for rough walls with a modification originally proposed
by Boeing. ADflow is specialized in optimizations and thus uses the
\textit{adjoint method} to compute the gradients needed in an efficient manner.
To make the rough modification available for optimizations, the changes have
been differentiated using Automatic Differentiation (AD).

    For verification, the implementation is compared against theory, the open
source CFD solver SU2 and experimental data of a \textit{flat plate a zero
incidence} for various surface roughnesses. The modified gradients are verified
using the complex step method.

    The results show that the implementation under-predicts the effect of
roughness. But the predicted shape of the effect seems correct. The computed
gradients only match to a relative tolerance of \num{1e-7} compared to complex
step. A relative tolerance of less than \num{1e-8} would be desirable with the
methods employed.


    \end{abstract}

    \tableofcontents\clearpage


    \chapter{Introduction}
    High fidelity optimization using gradient based approaches have become more
and more popular as they do not suffer from the \textit{curse of
dimensionality}. This allows them to have a much higher number of
design variables (dvs) compared to genetic optimization
approaches. This report describes how the SST turbulence model was modified to
work in ADflow for simulation and gradient computation.



\section{ADflow}
ADflow is an open-source multi-block\footnote{Overset meshes are also
possible.} Computational Fluid Dynamics (CFD) solver. It solves the Reynolds
Averaged Navier Stokes (RANS) equations to obtain the flow solution. It is
developed and maintained at the \textit{MDOLab} at the university of Michigan
and was based on a CFD solver for turbo machinery called \textit{sumb}. It has
later been adopted  for gradient based optimization by means of
\textit{Algorithmic Differentiation}\footnote{Thats what AD stands for in
ADflow.} and the \textit{adjoint method}. Most of ADflow is written in Fortran.
But it is interfaced in Python. This means, the heavy lifting is done in a fast
language, but the regular user has the benefits of an object-oriented high
level interpreter language.

 
In optimization, a lot of simulations are necessary until an optimal design is
found. It is also highly important to always get an objective value for each
design, even if, or especially when, it is unphysical. Otherwise the optimizer
does not know how bad the current design is. To cater those concerns, ADflow
employs some highly efficient and robust NK\footnote{NK stands for the
Newton–Krylov method.} and ANK\footnote{ANK is an approximated Newton–Krylov
method.} solvers. Those can achieve machine-precision convergence, even for
aircraft configurations at an angle of attack of 90\degree \cite{Mader2020a}
\cite{Kenway2019a} \cite{Yildirim2019b}.

When the solver was still called sumb, various turbulence models such as
Spalart-Allmaras, Spalart-Allmaras with Edwards Modification, $k$ - $\omega$
Wilox, $k$ - $\omega$ Wilox modified, $k$ - $\tau$, v2-f and Menter SST were
implemented. The subsequent modification for optimizations changed the
structure dramatically and only the SA model was carried over.


\section{Goals}
As stated above, the legacy code for SST is still available but does not really
work anymore. The goal for this project is to get it working for simulation and
optimization. The necessary sub-steps may be summarized as follows:

\begin{enumerate}
    \item Get the current SST model running using a legacy DADI-method.

    \item Modify it in such a way that it is automatically differentiate-able.

    \item Actually differentiate it through AD

    \item Make sure the partial AD derivatives are correct. 

    \item Get the NK/ANK Solver working.

    \item Test and verify the implementation

    \item Get the adjoint Solver working.

    \item Test and verify the modified gradients.
\end{enumerate}




\section{Code contributions}
Part of this project is a code contribution to ADflow and a setup of test
cases, both can be found on GitHub under those links:\\

\begin{tabular}{l l}
  ADflow Pull Request & \url{LINK_TO_PR} \\
  Test cases & \url{https://github.com/DavidAnderegg/SST_rough_testcases}
\end{tabular}


    \chapter{Theoretical Fundamentals}
    \section{Reynold's Averaged Navier Stokes (RANS).}
The \textit{Navier-Stokes} equations establish the connection between the
velocity ($U$), pressure ($P$), temperature ($T$), and density ($\rho$) of a
moving fluid. These equations, represented as partial differential equations,
are typically challenging to solve analytically. Consequently, numerical
methods become necessary. The physical properties are functions of four
variables: spatial coordinates (x, y, z), and time (t). To apply numerical
methods effectively, these properties must be discretized \cite{nasaNS}.

In flows with high Reynolds numbers, various eddies have significantly
different length and time scales. Properly capturing the smallest eddies,
essentially representing turbulence, would require an extremely fine mesh and
time step, rendering it impractical. To address this challenge, simplifications
are necessary: (1) considering a steady flow instead of unsteady, and (2)
accounting for turbulence in a stochastic manner. By adopting these approaches,
a single simulation (instead of one every x milliseconds) and a coarser grid
become sufficient to achieve meaningful results.


\subsubsection{Reynolds averaging}
In 1895, Osborne Reynolds introduced a solution that would later be referred to
as \textit{Reynolds Averaged Navier-Stokes}. The fundamental concept involves
dividing the velocity (along with other resolved physical properties) into two
components\cite{leschziner2015statistical} :

\begin{equation}
    U_{i} = \bar U_{i} + u_{i}^{\prime} \qquad
    P = \bar P + p^{\prime}
\end{equation}

\noindent The subscript $i$ represents all three spatial coordinates (x, y, z).
The mean velocity, denoted as $\bar U_{i}$, remains constant, while
$u_{i}^{\prime}$ represents the fluctuating component caused by turbulence,
which remains unresolved. The same notation applies to the pressure $P$.
Substituting these expressions into the incompressible Navier-Stokes equations
results in:

\begin{equation}
    \label{eq:incomp_RANS}
    \frac{\partial \rho \bar U_{i} \bar U_{j}}{\partial x_{j}} =
    \frac{\partial \bar P}{\partial x_{i}} +
    \frac{\partial}{\partial x_{j}} \nu (\frac{\partial 
    \bar U_{i}}{\partial x_{j}} +
    \frac{\partial \bar U_{j}}{\partial x_{i}}) -
    \frac{\partial}{\partial x_{j}} \rho u_{i}^{\prime} u_{j}^{\prime}
\end{equation}

\noindent The \textit{six} independent \textit{Reynolds-stresses} are
represented by $\rho u_{i}^{\prime} u_{j}^{\prime}$. It is important to note
that the fluctuating component for pressure, $p^{\prime}$, cancels out and does
not reappear. The Reynolds stresses can be denoted using tensor notation:

\begin{equation}
    \rho \bar u_{i} \bar u_{j} = \rho
    \begin{pmatrix}
        u_{1}^{\prime 2}              & u_{1}^{\prime} u_{2}^{\prime} & 
        u_{1}^{\prime} u_{3}^{\prime} \\

        u_{2}^{\prime} u_{1}^{\prime} & u_{2}^{\prime 2}              & 
        u_{2}^{\prime} u_{3}^{\prime} \\

        u_{3}^{\prime} u_{1}^{\prime} & u_{3}^{\prime} u_{2}^{\prime} & 
        u_{3}^{\prime 2}
    \end{pmatrix}
\end{equation}

\noindent Equation \ref{eq:incomp_RANS} is complemented by the Reynolds-averaged
mass-conservation equation:

\begin{equation}
    \frac{\partial \rho \bar U_{j}}{\partial x_{j}} = 0
\end{equation}


\subsubsection{Turbulence model}
To determine the unknown Reynolds stresses, a \textit{turbulence model} is
employed. Among several approaches, the two most commonly used are \textit{Eddy
viscosity} models and \textit{Reynolds stress transport} models. The SST model
belongs to the former category, and therefore, it will be elaborated on in more
detail.


\paragraph{Eddy viscosity models}
These kind of models depend on the \textit{Boussinesq-assumption} which says
that the effect of turbulence is similar to that of an increased viscosity.
Thus, it introduces the \textit{eddy viscosity} $\mu_{t}$. After some equation
mangling one may calculate the Reynolds stresses from the eddy viscosity as
follows:

\begin{equation}
    - \rho u_{i}^{\prime} u_{j}^{\prime} =
    \nu_{t} (\frac{\partial \bar U_{i}}{\partial x_{j}} +
    \frac{\partial \bar U_{j}}{\partial x_{i}}) -
    \frac{2}{3} \delta_{ij} \rho k
    \label{eq:boussinesq}
\end{equation}

\noindent Where

\begin{align*}
    \delta_{ij} = \; &0 \qquad \text{for} \qquad i \neq j \\
    &1 \qquad \text{for} \qquad i = j
\end{align*}

\noindent and $k$ is the \textit{turbulent kinetic energy}. Thus the calculation
of the six Reynolds-stresses has reduced to calculating $\nu_{t}$ and $k$
\cite{leschziner2015statistical}.




\subsection{$k$ - $\epsilon$ model}




\subsection{$k$ - $\omega$ model}




\subsection{$k$ - $\omega$ SST model}


\section{Grid Convergence}
When discretizing a partial differential equation and solving it numerically,
an error is introduced. It may be decreased through a finer mesh or a higher
order method. To demonstrate that the method approaches the exact solution,
finder and finer grids are used. This process is called a \textit{grid
refinement study} or \textit{mesh convergence}. For a given grid, the grid
spacing is:

\begin{equation}
  h = N^{-1/d}
\end{equation}

\noindent Where $N$ is the number of cells and $d$ is the dimension of the
problem.
For ADflow, the expected rate of convergence is $p=2$. But in reality, this
might not be the case. For three grids, the actual rate may be calculated as
follows:

\begin{equation}
  \hat p = ln(\frac{f_{L2} - f_{L1}}{f_{L1} - f_{L0}}) / ln(r)
  \label{eq:conv_rate}
\end{equation}

\noindent Where $f$ is the function of interest (e.g. $c_{d}$) and the subscript
tells the grid used. $L0$ is the finest grid and $L2$ the coarsest. The
parameter $r$ is the grid refinement ratio \cite{grid_refinement}.


    \chapter{Methods}
    \section{Introductory thoughts}
To start, a small introduction to ADflow's solvers, its history and the initial
state of SST is given.

In the beginning, this solver was called \textit{Standford University
Multiblock (sumb)}. It was intended as a multiblock solver for turbomachinery.
Later on, it was extended to be used in optimizations. For this, an adjoint
solver was needed. As explained in section \ref{sec:gradient_computation}, the
adjoint method needs partial derivatives which are obtained through means of
finite differences and/or automatic differentiation. For the AD part, a tool
called \textit{tapenade}
\footnote{\url{http://www-tapenade.inria.fr:8080/tapenade/index.jsp}} is used.
It automatically differentiates FORTRAN source code. But to get it working, a
lot of the initial structure had to be changed.




\subsection{Flow Solvers}
ADflow has three different solvers available to solve the RANS equations:
\textit{multigrid (MG)}, \textit{Newton-Krylov (NK)} and \textit{Approximate
Newton-Krylov (ANK)}. It is possible to switch between
the different solvers during a solution run. This allows to use each Solver
when it is most efficient: initiate the simulation using multigrid, once a
certain level of convergence is reached, engage the ANK solver an finally
converge the last couple order of magnitudes using the NK Solver.
\footnote{Please note that the ANK solver by itself is sufficient as a startup
strategy and MG is not necessarily needed.}

Multigrid is the baseline solver that was implemented first. It uses either the
Runge-Kutta (RK), or the Diagonalized Diagonally-Dominant Alternating Direction
Implicit (D3ADI) algorithm as a smoother. The turbulence model is solved in a
decoupled manner and using the Diagonalized Alternating Direction Implicit
(DADI) method.

The Newton-Krylov solver solves the nonlinear system of governing equations by
simply using the Newton's method. To solve this linear system, the GMRES
algorithm is used. The turbulence variables are solved in a coupled manner and
thus no other solvers are needed. This method is equivalent to using Euler's
method with an infinite time step. It is most efficient when the solution is
already at the final stages of convergence. If it is used in the early stages,
it most likely stalls.

The Approximate Newton-Krylov is similar to the NK solver in that it also uses
Euler's method. But its time step is adjustable. At the beginning of the run,
it is quite low and thus increases the stability. This allows it to be used
as a startup strategy where the NK solver wouls most certainly fail. The
lower the residual norm is, the more the time step is increased. And as such, it
approaches Newtons' method. It is subdivided into three different sub-solvers:
\textit{First order ANK (ANK)}, \textit{Second Order ANK (SANK)} and
\textit{Coupled ANK (CANK)}. Please note, the combination of both
\textit{Coupled Second-Order ANK (CSANK)} is also possible.

In its base configuration (ANK) uses a first-order routine for the residual
Jacobian and thus only affects convergence, but not the solution accuracy. Once a user defined level of
convergence is reached, the solver switches to an exact Jacobian formulation
(SANK). In these two stages, the turbulence model is solved in a decoupled
manner. Once can choose between a second turbulence-only-ANK solver or the
above mentioned DADI method. It is done this way because the turbulence models
in RANS simulations are notoriously difficult to converge. Once again, when the
residual norm reaches a user-defied level, the coupled mode is enganged. In
that state, only one ANK solver remains which solve the flow and turbulence
variables simultaneously. This helps to improve the convergence in the later
stages. \cite{adflow_solvers}




\subsection{Adjoint Solver}




\subsection{Initial state of SST}
When sumb was developed initially, a multitude of different turbulence models
were implemented such as Spalart-Allmaras (SA) or SST. When the code was
overhauled for optimization, only the SA model was carried over and
differentiated. At that point SST would throw NaNs\footnote{Not a Number} and
crash. Lately, this was fixed to a point where the DADI turbulence solver would
work \footnote{See pull request:
\url{https://github.com/mdolab/adflow/pull/107}}. To summarize, before this
project started, the code for SST was there and could be solved using ANK/SANK
and the decoupled DADI turbulence solver. But nothing was differentiated which
means, the adjoint, NK and coupled ANK solvers were not usable.








\section{Needed changes}
ADflow can parallelize the computation over multiple cpus and computers. It
does this by splitting up the computational domain into blocks. These blocks
may live on different cpus or computers. As those blocks do not live in
isolation and do depended on each other, adjacent blocks need to exchange
information. ADflow uses the \textit{halo cell} approach for this. The idea is
to have imaginary ghost (halo) cells around each block. Then they are filled
with the values from the adjacent blocks. Figure \ref{fig:halo_cells} shows
this idea. One difference to the picture is that ADflow uses two layers of halo
cells.

\begin{figure}[H] \centering
\includegraphics[width=0.7\textwidth]{halo_cells}
    \caption{A block split in 4 (left) and its corresponding halo cells (right)
            \cite{cfd_halo}.}
    \label{fig:halo_cells}
\end{figure}




\subsection{Halo exchange and AD}
This halo exchange is straight forward when one does not care about automatic
differentiation. But we do, and as such, some things need to be considered. The
exchanges is performed using the \textit{Message Passing Interface (MPI)}.
Unfortunately, tapenade can not handle MPI calls. To make this work, the
differentiated code is divided into parts: The \textbf{math heavy} part is
differentiated using tapenade. The \textbf{remaining part} is
hand-differentiated. This is not a big problem as this part mostly consists of
calling the AD part and performing communication. 

The turbulence model is considered a math heavy part that is automatically
differentiated using tapenade. For SA, this was straight forward as there is no
communication going on. But SST has one special case: The blending function
$\mathbf{F_1}$ (equation \ref{eq:f1}). The problem is, it is not a normal
turbulence working variable (such as $k$ or $\omega$) and thus is not exchanged
using existing infrastructure. The developers of sumb fixed this by exchanging
it manually after computing it. This is done in the model itself and was no
problem because nobody intended to AD it. \\

Now the question is: Why was it done this way and how do we get rid of this
intermediate communication?

The first part of the questions might be explained by fact that sumb was
initially developed in the early 2000's. Back then, computing power was more
expensive and computing the same $F_1$ value on different blocks was
wasteful. The cost of communication was negligible in comparison. This did not
really change, but computing power became a lot cheaper. So if it makes AD
easier, it is a good trade to get rid of the communication.

\section{Algorithmic/Automatic Differentiation}

\section{Verification}

\subsection{Testcases}

\subsubsection{NACA 0012}
\subsubsection{RAE 2822}
\subsubsection{Flatplate}
\subsubsection{2D bump}
\subsubsection{3D wing}


\subsection{Partial derivatives}

\subsection{Total derivatives}





    \chapter{Results}
    \section{Solver Convergence}
\subsubsection{NACA0012}
Figure \ref{fig:sc_NACA0012} shows the solver convergence for the NACA0012
testcase. The baseline implementation and the final state when writing this
report (modified) is shown. Both states were run once with 1 and 6 cpus.

\begin{figure}[H] \centering
    \includegraphics[width=1.0\textwidth]{plots/sc_NACA0012}
    \caption{Convergence history for NACA0012 testcase.}
    \label{fig:sc_NACA0012}
\end{figure}

\noindent Lets take a look at the baseline implementation first. It was
obtained using the ANK solver and a decoupled DADI solver for the turbulence
model. For the turbulence model, a total of 20 sub iterations were used. When
looking at the functions values (top row), it can be seen that 1 and 6 cpus
reach the same value. But 6 cpus need more iterations. It is not quite clear
what causes this, but it is a known phenomenon for low cpu counts and vanishes
when this number is increased. Production runs usually require tens or even
hundrets of cpus and thus this is not considered detrimental.


Now, lets look at the modified curves. Here, SST was differentiated and the
turbulence ANK and fully coupled CANK solvers are available. Annectdotal
evidence suggest SST is highly non-linear. This is especially true for the
inital stages of convergence. Due to this\footnote{The author believes the ANK
solver does some finite-differencing for some terms under the hood.}, the
turbulence DADI solver is way more efficient early on. Thus, at the beginning,
the regular ANK solver with decoupled DADI was used. But once a relative
convergence of 1e-6 is reached, the second order coupled ANK (CSANK) is
engaged. Once it gets traction, it exhibits almost Newton-like convergence. The
number of cpus does not really affect the number of iterations needed. It is
also obvious that the modified version approaches the same function values as
the baseline implementation. 




\subsubsection{RAE2822}
Figure \ref{fig:sc_RAE2822} shows a similar convergence plot for the RAE2822
testcase. Once again, the baseline and modified version with each 1 and 6 cpus
is plotted. It is important to note that this case is somewhat hard as it lies
in the transsonic regimes where shocks appear. But at the same time, it is even
coarser than the naca case which makes it hard to resolve them properly.

\begin{figure}[H] \centering
    \includegraphics[width=1.0\textwidth]{plots/sc_RAE2822}
    \caption{Convergence history for RAE2822 testcase.}
    \label{fig:sc_RAE2822}
\end{figure}

\noindent First, lets glance ath the baseline. This has also been obtained
using the ANK solver for the flow variables and the DADI solver for the
decoupled turbulence variables. A similar pattern to the NACA
testcase appears: 1 cpu takes only half the iterations of what 6 cpus need.
But, the converged values are the same. When comparing the general line pattern
to the NACA testcase, it appears to be more 'wiggly' here. The author belieaves
this is due to the coarse mesh and transsonic regime. This probably increases
the sensitivity to the CFL number. During convergence, the ANK solver increases
the CFL number based on the current relative convergence. But the high
sensitivity makes the solver unstable and i starts to diverge. Once this is
detected, the CFL number is lowered again. Plese note that it is not as simples
as it seems here because the convergence of the linear system for the newton
step also influences the CFL number. The author just wants to stress that
probably some form of coupling causes the unsteady behavior. This should be
avoidable, but maybe more tuning or even a change to the CFL-ramping algorithm
is needed.

When looking at the modified version, a similar picture to NACA emerges. The
strategy was the same, first use ANK with DADI and once a relative convergence
of 1e-6 is reached, the CSANK solver is enganged which shows almost Newton-type
convergence. Although the contrast is not as big. But it also has to be noted
that that the before mentioned CFL dependence played a role here and some
parameters had to be clipped to increase robustness at the cost of convergence.





\section{Grid Convergence}
Figure \ref{fig:gc_2d_bump} shows the grid convergence for the 2d bump testcase
compared to data from the \textit{CFL3D} and \textit{FUN3D} CFD solvers. At
first glace, ADflow seems to be in the right bulk part but does not completely
agree with the reference. This difference may be explained through slightly
different model formulations. The turbulence production term used in ADflow is
called \textit{strain}. The author tried to converge the same case using the
\textit{vorticity} formulation but was unable to so. The reference data was
obtained using the \textit{SSTm} formulation. It is also possible that ADflow
uses a slightly different version.

\begin{figure}[H] \centering
    \includegraphics[width=1.0\textwidth]{plots/gc_2d_bump_nan-fix}
    \caption{Grid convergence for 2D bump. Reference data is from
    \cite{nasatmr}.}
    \label{fig:gc_2d_bump}
\end{figure}

\noindent But ignoring the discrepancies, the values seem to approach a certain
value which is desired.








\section{Partial derivatives}
\subsubsection{Forward mode}
To verify the forward partial derivatives, the 3D test case is first converged
to a relative tolerance of $1e-14$.\footnote{In theory, the partials should be
accurate regardless of the current convergence. But it is more valuable when
one can show that they are accurate for a converged state as this is what we
are after. } Once this is the case, the partials are compared against finite
difference and complex step. Anecdotal evidence suggest SST is highly
non-linear which is reflected in the fact that the FD partials are quite off
compared to AD. Table \ref{tab:partials_forward} lists the derivatives with the
relative accuracy compared to CS (stepsize was $1e-40$). Please not that the
tolerances had to be lifted partly to $1-e9$ instead of $1-e10$. This also
indicates that SST is highly non-linear.

\begin{table}[H]
    \centering
    \begin{tabular}{l r}
        \toprule
        Derivative                          & Relative tolerance \\
        \toprule
        $\partial R / \partial u$           & $\leq 1e-9$ \\
        $\partial f / \partial u$           & $\leq 1e-9$ \\
        $\partial F / \partial u$           & $\leq 1e-9$ \\
        \midrule
        $\partial R / \partial x_{geo}$     & $\leq 1e-9$ \\
        $\partial f / \partial x_{geo}$     & $\leq 1e-9$ \\
        $\partial F / \partial x_{geo}$     & $\leq 1e-9$ \\
        \midrule
        $\partial R / \partial x_{aero}$    & $\leq 1e-10$ \\
        $\partial f / \partial x_{aero}$    & $\leq 1e-10$ \\
        $\partial F / \partial x_{aero}$    & $\leq 1e-10$ \\
        \bottomrule
    \end{tabular}
    \caption{Relative accuracy of forward AD partials compared to CS.}
    \label{tab:partials_forward}
\end{table}


\subsubsection{Reverse mode}
To verify the backwards partials, a dot-product test was performed. Table
\ref{tab:partials_dotproduct_test} lists the tests  with the relative tolerance
it was passed. The only test that did not pass was $u \rightarrow F$. As $F$
are the nodal forces, those derivatives are only needed for aerostructural
optimization. The reason for failing is probably the fact, that those routines
buffer some values without recomputing. The changes to the wall distance
probably require changes to those routines as well. Because this was not done,
the test fails.

\begin{table}[H]
    \centering
    \begin{tabular}{l r}
        \toprule
        Dot product test                     & Relative tolerance \\
        \toprule
        $u \rightarrow R$                   & $\leq 1e-10$ \\
        $u \rightarrow F$                   & $\textcolor{red}{= 2e-6}$ \\
        \midrule
        $x_{geo} \rightarrow R$             & $\leq 1e-9$ \\
        $x_{geo} \rightarrow F$             & $\leq 1e-10$ \\
        \midrule
        $(u, x_{geo}) \rightarrow (du, F)$  & $\leq 1e-10$ \\
        \bottomrule
    \end{tabular}
    \caption{Relative accuracy of dot product test between forwards and
    backwards partial derivatives.}
    \label{tab:partials_dotproduct_test}
\end{table}


\subsubsection{Reverse\_fast mode}
The reverse\_fast partials are simply compared to the forwards routines. This
is possible because the first one is a subset of the former one. Table
\ref{tab:partials_fast} lists the relative accuracy. It obviously does not
agree at all. This is due to time constraints as the author could not proceed
to improve it further. But the fact that it yields a number and does not simply
crash is already an achievement.

\begin{table}[H]
    \centering
    \begin{tabular}{l r}
        \toprule
        backwards vs backwards\_fast        & Relative tolerance \\
        \toprule
        $u$                                 & $\textcolor{red}{= 5.7e4}$ \\
        \bottomrule
    \end{tabular}
    \caption{Relative accuracy of backwards\_fast routines compared to
    backwards.}
    \label{tab:partials_fast}
\end{table}







\section{Total derivatives}
As described in sec. XXX, the total derivatives were verified by comparing them
to complex step. As described in sec. XXX, ADflow may assemble the adjoint
system using either the forwards partials or the reverse\_fast partials. Table
\ref{tab:total_derivs} lists the relative difference of various methods
against complex step with a stepsize of $1e-200$.

\begin{table}[H]
    \centering
    \begin{tabular}{l r r r r}
        \toprule
        Name        & cmplx. ($h=1e-40$)   & adj. frwd. 1cpu           & 
            adj. frwd. 6 cpus         & adj. rev. fast 6 cpus    \\
        \toprule
        alpha       & -1.3e-09             & -5.3e-06                  & 
            -5.3e-06                  & \textcolor{red}{-8.0e-03} \\
        mach        & -6.1e-08             &  2.3e-05                  &  
            2.3e-05                  & \textcolor{red}{ 2.3e-01} \\
        \midrule
        span \#0    & -2.0e-09             &\textcolor{red}{ 4.0e-04}  &
            \textcolor{red}{-9.0e-04}  & \textcolor{red}{-1.0e-02} \\        
        twist \#0   &  9.3e-09             &\textcolor{red}{-1.0e-03}  &
            9.6e-05                    & \textcolor{red}{-1.4e-02} \\        
        shape \#0   & -5.3e-07             &\textcolor{red}{ 5.8e-02}  &
            \textcolor{red}{ 5.6e-02}  & \textcolor{red}{ 1.2e-00} \\         
        \bottomrule
    \end{tabular}
    \caption{Relative difference of gradients compared to compplex step
    (stepsize = 1e-200). The function of interest is $C_l$. When the DVs are a
    vector (e.g span), only the first variable is listed.}
    \label{tab:total_derivs}
\end{table}

\noindent First, lets look at the forward routines with 1 and 6 cpus. The first
two variables are \textit{aerodynamic}, meaning do not control the geometry. It
appears that they stay the same on 1 or 6 cpus. Their relative accuracy is
approx 1e-5. This is relatively low, but may be explained through the high
non-linearity of SST. It is also important to realize that those are the first
tries. This means, there will probably be better options and/or methods and
thus a higher accuracy should be expected in the future. When looking at the
\textit{geometric} derivatives, it becomes clear that they differ with the
number of cpus. Also the are less accurate than the geometric ones. This
probably indicates a problem with modification done to the wall distance
exchange routines. This should be fixable with only a bit more time.

As already indicated with the partial reverse\_fast routines, the total
derivatives are also completely wrong. But once again, the fact that a number
was obtainable is already an achievement. This should also be fixable and only
needs a bit more time.








\section{Summary}
The results show that a prototype state was achived. SST does converge using
the coupled ANK solver and also gradients can be obtained. But of course, there
are still some bugs present. Some other results were more anecdotal and could
not be proven properly. The following list shows the insights gained through
this project.

\begin{itemize}
    \item SST is probably highly non-linear.

    \item ANK does work using the AD preconditioner. But sometimes NaNs appear.

    \item ANK does not work using FD preconditioner (probably due high
        non-lineary of SST).

    \item Physicality check for ANK needs to be adjusted to SST.

    \item Prototype adjoint is running.

    \item Aerodynamic derivatives using forward routines are accurate.

    \item Geometric derivatives using forward routines are not accurate.

    \item There is probably a bug in the halo exchange of the wall distance in
        the backwards routines.

    \item Derivatives using reverse\_fast routines are not accurate at all.

    \item SST appears to be implemented correctly.
\end{itemize}



    \chapter{Conclusion}
    To conclude, this project has made significant strides in enhancing the SST
turbulence model for optimization purposes within ADflow. The following key
points highlight the project's achievements:

First, the distance to the nearest wall had to be extended for halo cells. This
was necessary in order to get rid of some intermediate MPI communication in the
$F_1$ blending function of SST. It was important to get rid of this
communication as it prevents automatic differentiation.

The implementation of Automatic Differentiation (AD) for the SST turbulence
model was a major milestone. This implementation facilitated the efficient
computation of partial derivatives, which are essential for optimization tasks
in ADflow. But also the regular solvers like \textit{Approximate Netwon-Krylov
(ANK)} and \textit{Newton-Krylov (NK)} profit from this as they now have access
to an AD-preconditioner.

It has been shown that the changes did not change the legacy implementation of
SST. Using testcases from NASA's \textit{Turbulence Modeling Resource (TMR)},
it has also been shown that the results obtained mostly agree to reference data.
Although there is still some uncertainty because a different production term for
the turbulence had to be used in order do achieve convergence. It was also not
possible to obtain results for the \textit{flatplate} testcase. The author is
confident that a bit more time would adress this shortcomings.

It is even possible to obtain total gradients using the adjoint method when
assembling the state residual matrix using forward routines. The aerodynamic
design variables appear to be correct, although to relatively low tolerance.
This my be explained through the high non-linearity of SST. It is also
important to note that the geometric design variables seem to be wrong.
Additionally, they appear to depend on the number of cpus used. This most
likely indicates a problem with the extension of the wall distance to halo
cells.

When trying to solve the adjoint system using the reverse\_fast routines, on
may get a number. But it is completely wrong. But the author still considers
this an achievement because ADflow does not simply crash.

In order to maintain the quality and integrity of ADflow, regression test are
used. This project did extended them for the SST turbulence model. But because
the implementation of SST is only at a prototype stage, the tests are as well.

\section{Comparison to Goals}
When looking at the goals stated in the introduction (section \ref{sec:goals}),
most of them have been partially achieved. The following lines take a look at
the goals that were missed or only partially fulfilled.

\begin{enumerate}

    \item[4.] Make sure the partial AD derivatives are correct. 
        \begin{itemize}
            \item This has almost been achieved. Only the backwards mode for
                the Forces is wrong. Also, the backwards\_fast mode is not
                correct.
        \end{itemize}

    \item[5.] Get the NK/ANK Solver working.
        \begin{itemize}
            \item Apart from the finite differences pre-conditioner, this has
                been achieved.

            \item The physicality check probably also needs some adjustment for
                SST.
        \end{itemize}

    \item[6.] Test and verify the implementation
        \begin{itemize}
            \item This has been partially achieved. It would be great if the
                flatplate testcase would have worked. Also the different
                production terms for SST need to be tested more.
        \end{itemize}

    \item[7.] Get the adjoint Solver working.
        \begin{itemize}
            \item Has been achieved in the sense that it does not crash 
        \end{itemize}

    \item[8.] Test and verify the total gradients.
        \begin{itemize}
            \item This has been partially achieved. The aerodynamic design
                variables appear to be correct when using forward AD for the
                state residual matrix. But the aerodynamic variables are
                slightly off. 

            \item The gradients are completely off when using the
                backwards\_fast routines.
        \end{itemize}
\end{enumerate}

To conclude, SST has now achieved a prototype state where it is almost ready
for optimization. There are still some bugs present, but it is clear where they
are and mostly by what they are caused. The only thing that prevented the
author from fixing them was a lack of time. As such, one has to realize that
getting SST fully working for optimization is a big task. Thus, the outcome of
this project should be considered a success.



\section{Outlook}
This report stated openly and clearly where bugs are present and by what they
are most likely caused. As fixing those bugs and getting SST ready for
production is probably the most valuable target of a proceeding project, the
steps that come next should be clear. But still, the following line lists the
most prominent points in random order:

\begin{itemize}
    \item Adapt the physicality check in ANK-turb for SST

    \item Get the finite difference preconditioner working in ANK for SST.
        Maybe only the step size needs to be changed.

    \item Make sure the partials are correctly summed up in the reverse
        routines that exchange the wall distance in halos.

    \item Take a closer look at what is going on with total derivatives
        obtained through complex step. The results have shown that there is a
        slight difference between a step size of $1e-40$ and $1e-200$. This
        should not be the case.
        
    \item Make sure all the gradients are correct when solving the adjoit system
        using the forward AD routines.

    \item Fix the partial backwards\_fast derivatives.

    \item Make sure the gradients are correct when using the backwards\_fast
        routines.

    \item Optimize an airfoil using SST and compare it to the same optimization
        using SA.
\end{itemize}



    \listoffigures\clearpage
    \listoftables\clearpage


    \printbibliography

\end{document}
