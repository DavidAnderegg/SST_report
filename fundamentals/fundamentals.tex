\section{Reynold's Averaged Navier Stokes (RANS)}
The \textit{Navier-Stokes} equations establish the connection between the velocity ($U$),
pressure ($P$), temperature ($T$), and density ($\rho$) of a moving fluid. These
equations, represented as partial differential equations, are typically
challenging to solve analytically. Consequently, numerical methods become
necessary. The physical properties are functions of four variables: spatial
coordinates (x, y, z), and time (t). To apply numerical methods effectively,
these properties must be discretized \cite{nasaNS}.

In flows with high Reynolds numbers, various eddies have significantly
different length and time scales. Properly capturing the smallest eddies,
essentially representing turbulence, would require an extremely fine mesh and
timestep, rendering it impractical. To address this challenge, simplifications
are necessary: (1) considering a steady flow instead of unsteady, and (2)
accounting for turbulence in a stochastic manner. By adopting these approaches,
a single simulation (instead of one every x miliseconds) and a coarser grid
become sufficient to achieve meaningful results.


\subsubsection{Reynolds averaging}
In 1895, Osborne Reynolds introduced a solution that would later be referred to
as \textit{Reynolds Averaged Navier-Stokes}. The fundamental concept involves dividing
the velocity (along with other resolved physical properties) into two
components\cite{leschziner2015statistical} :

\begin{equation}
  U_{i} = \bar U_{i} + u_{i}^{\prime} \qquad
  P = \bar P + p^{\prime}
\end{equation}

\noindent The subscript $i$ represents all three spatial coordinates (x, y, z).
The mean velocity, denoted as $\bar U_{i}$, remains constant, while
$u_{i}^{\prime}$represents the fluctuating component caused by turbulence,
which remains unresolved. The same notation applies to the pressure $P$.
Substituting these expressions into the incompressible Navier-Stokes equations
results in:

\begin{equation}
  \label{eq:incomp_RANS}
  \frac{\partial \rho \bar U_{i} \bar U_{j}}{\partial x_{j}} =
  \frac{\partial \bar P}{\partial x_{i}} +
  \frac{\partial}{\partial x_{j}} \nu (\frac{\partial \bar U_{i}}{\partial x_{j}} +
  \frac{\partial \bar U_{j}}{\partial x_{i}}) -
  \frac{\partial}{\partial x_{j}} \rho u_{i}^{\prime} u_{j}^{\prime}
\end{equation}

\noindent The \textit{six} independent \textit{Reynolds-stresses} are
represented by $\rho u_{i}^{\prime} u_{j}^{\prime}$. It is important to note
that the fluctuating component for pressure, $p^{\prime}$, cancels out and does
not reappear. The Reynolds stresses can be denoted using tensor notation:

\begin{equation}
  \rho \bar u_{i} \bar u_{j} = \rho
  \begin{pmatrix}
    u_{1}^{\prime 2}              & u_{1}^{\prime} u_{2}^{\prime} & u_{1}^{\prime} u_{3}^{\prime} \\
    u_{2}^{\prime} u_{1}^{\prime} & u_{2}^{\prime 2}              & u_{2}^{\prime} u_{3}^{\prime} \\
    u_{3}^{\prime} u_{1}^{\prime} & u_{3}^{\prime} u_{2}^{\prime} & u_{3}^{\prime 2}
  \end{pmatrix}
\end{equation}

\noindent Equation \ref{eq:incomp_RANS} is complemented by the Reynolds-averaged
mass-conservation equation:

\begin{equation}
  \frac{\partial \rho \bar U_{j}}{\partial x_{j}} = 0
\end{equation}


\subsubsection{Turbulence model}
To determine the unknown Reynolds stresses, a \textit{turbulence model} is
employed. Among several approaches, the two most commonly used are \textit{Eddy
viscosity} models and \textit{Reynolds stress transport} models. The SST model
belongs to the former category, and therefore, it will be elaborated on in more
detail.


\paragraph{Eddy viscosity models}
These kind of models depend on the \textit{Boussinesq-assumption} which says
that the effect of turbulence is similar to that of an increased viscosity. Thus
it introduces the \textit{eddy viscosity} $\mu_{t}$. After some equation
mangling one may calculate the Reynolds stresses from the eddy viscosity as
follows:

\begin{equation}
  - \rho u_{i}^{\prime} u_{j}^{\prime} =
  \nu_{t} (\frac{\partial \bar U_{i}}{\partial x_{j}} +
  \frac{\partial \bar U_{j}}{\partial x_{i}}) -
  \frac{2}{3} \delta_{ij} \rho k
  \label{eq:boussinesq}
\end{equation}

\noindent Where

\begin{align*}
  \delta_{ij} = \; &0 \qquad \text{for} \qquad i \neq j \\
                   &1 \qquad \text{for} \qquad i = j
\end{align*}

\noindent and $k$ is the \textit{turbulent kinetic energy}. Thus the calculation
of the six Reynolds-stresses has reduced to calculating $\nu_{t}$ and $k$
\cite{leschziner2015statistical}.




\subsection{$k$ - $\epsilon$ model}




\subsection{$k$ - $\omega$ model}




\subsection{$k$ - $\omega$ SST model}


